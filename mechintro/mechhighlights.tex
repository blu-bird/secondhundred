\section*{Mechanics Highlights}
\addcontentsline{toc}{section}{\protect\numberline{}Mechanics Highlights}
Some of my personal favorite mechanics problems. These are hard problems, but I strongly encourage you to give most of these a try (except problem 1), because they're good problems. If you feel like you've attempted a problem several times and come up with an incorrect answer, I'd be happy to discuss these with you and go over them. \\
\textbf{Problems:}\\
1. (?? $\bigstar$) Find as many mistakes/inaccuracies as you can find in the Mechanics section. \\
2. (4 $\bigstar$) A solid bowling ball of mass $M$ and radius $R$ has an initial clockwise angular velocity $\omega_0$ down the lane and an initial speed of the the center of mass of the ball $s = \frac{R\omega_0}{3}$ just after its release. The coefficient of kinetic friction is $\mu_k$. Show that the final angular momentum of the ball about its initial point of contact with the floor is $\frac{11}{15}MR^2 \omega_0$.\\
3. (3 $\bigstar$) Consider a solid cylinder that has mass $M$ and radius $R$ to which a second solid cylinder that has mass $m$ and radius $r$ is attached. These cylinders share an axis of symmetry. A string is wound about the smaller cylinder. The larger cylinder rests on a horizontal surface. The coefficient of static friction between the larger cylinder and the surface is $\mu_k$. If a light tension is applied to the string in the vertical direction, the cylinder will roll to the left; if the tension is applied with the string horizontally to the right, the cylinder rolls to the right. Show that the angle between the string and the horizontal that will allow the cylinder to remain stationary when a
light tension is applied to the string is $\theta = \arccos \frac{r}{R}$. \\
4. (3 $\bigstar$) A boy is initially seated on the top of a hemispherical ice mound of radius $R$. He begins to slide down the ice, with a negligible initial speed. Approximate the ice as being frictionless. Show that the height where the boy loses contact with the ice is $\frac{2}{3}R$. \\
5. (4 $\bigstar$, $\spadesuit$) A rod of mass $M$ and length $L$ acts as a solid pendulum. Its pivot point is set a distance $d$ from its center of mass (in the center of the rod). Show that if the rod is set swinging (at small angles) and exhibits simple harmonic motion, the period of the motion can be maximized by setting $d = \frac{L}{2\sqrt{3}}$. Hint: Use blasphemy - $\sin \theta = \theta$ for small values of theta :P. \\
6. (5 $\bigstar$) A long rod of mass $M$ and length $L$ is at rest on a frictionless horizontal surface. A small blob of mass $M$ moving at a speed $s$ strikes the rod at one of its ends, and sticks to the rod. Show that the ratio of the total kinetic energy in the system before the collision to the total kinetic energy after the collision is $\frac{4}{5}$. \\
7. (4 $\bigstar$) A uniform rod with mass $M$ and length $L$ turns without friction on a fixed pivot at the end of the rod. The rod is released from rest from a horizontal position. The rod swings down, collides with a small dense block, also with mass $M$, and the block sticks to the end of the rod. Show that the maximum angular displacement of the rod from the vertical after the collision is $\theta = \arccos\left(\frac{11}{12}\right)$.\\
8. (5 $\bigstar$, $\spadesuit$) Two small blocks with masses $M$ and $2M$ are attached to a spring with negligible mass, spring constant $k$, and natural length $L$. The blocks and spring rest on a frictionless horizontal surface with the lighter block in contact with a wall located at $x=0$. The system is released from rest at time $t=0$ with the spring compressed to a length $\frac{L}{2}$. Show that the maximum length of the spring after the lighter block leaves the wall is $\left(1 + \frac{1}{2\sqrt{3}}\right) L$ and the location of the center of mass of the two-block system when this occurs is $\frac{\pi\sqrt{3}+2}{3}L$.
\pagebreak