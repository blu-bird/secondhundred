\setcounter{section}{-1}
\section{Introduction}
Well, here we go again. 

One thing's for certain: I have a lot of time on my hands, and everyone does now because of quarantine. At the time of writing, we are all holed up in our homes, responsibly social distancing but also doing not much of anything. 

A couple of years ago, I wrote ``Physics in One Hundred Pages,'' a guide to physics at the AP level, chronicling the basics of mechanics and electromagnetism. Since then, I've learned so much more about other fields of physics and I really wanted to do a continuation of this with all of this extra time. Therefore, this is a new (ongoing) project dedicated to building a repository of physics knowledge at TJ that future years can read quickly, pick up, and distribute. We hope to do this via understandable and accessible entry points, while not ditching a rigorous discussion of the subject. This includes a revamping of the old Hundred Pages, lectures and handouts given at TJ's Physics Team for their A Team during the 2019-2020 school year, and other assorted transcriptions of lectures from classes taught at TJ (most notably from the Quantum/Electrodynamics class during the 2018-2019 school year). 

As one of the principal lecturers of TJ Physics Team's A Team for the past three-quarters of a year, I have formulated some strong opinions on what students should learn in order to round out their physics knowledge, and the contents of this paper reflect that. I also believe that math should be as low as a barrier to entry as possible, so throughout this handout I have chosen to sprinkle in chapters dedicated to relevant mathematics that is used in subsequent chapters. We've arranged the chapters in an order that we think makes sense 

Exercises that I feel are interesting, rewarding, tricky, or otherwise difficult are included, and I've tried to rate the difficulty of each problem from 1-5 stars ($\bigstar$), with more stars implying a harder problem. Obviously, no rating system can be fully objective, but I hope it helps to 
As I am not clever to come up with some of these problems on my own, most have been compiled from a variety of other sources, which will be noted when possible. 

I hope to be clear yet concise, and if anything seems unclear, feel free to contact me through Facebook Messenger, e-mail, or in person. I'm generally willing to help whenever I can.

I have a lot of people to thank for helping me with this project. I'd like to thank Ajit Kadaveru, Jason Chen, Sohom Paul, Jeffery Yu, and Aaditya Singh for helping me through physics when I took the course, editing numerous drafts of the original Hundred Pages, and for help with LaTeX and typesetting everything you see here nicely. A big thank you to the inaugural Quantum/Electrodynamics class under Dr. Osborne's tutelage, in particular Rubaiya Emran (my co-lecturer for this year). Thanks to Dr. John Dell and Dr. Jonathan Osborne, who have taught me a lot of what I know now and have instilled in me a passion for physics. And finally, thanks go out to all of you who read this, who make my work worth something! Your viewing and reading is greatly appreciated. 
\pagebreak