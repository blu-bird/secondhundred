\section{Perturbation Theory}
In quantum mechanics, we are typically concerned with finding the eigenvalues and eigenvectors of some arbitrary Hamiltonian. This is often very difficult, as the equations involved are in the majority of cases only solvable numerically. 

However, in many situations, the Hamiltonian is only slightly different from a Hamiltonian whose eigenvalues and eigenvectors are well-known. For example, if we apply a small electric or magnetic field to a hydrogen atom, the Hamiltonian will be a slight perturbation from the well-known hydrogen atom Hamiltonian. Additionally, the van der Waals forces between two hydrogen atoms, which are small in magnitude, could be treated as causing small perturbations on the hydrogen atom Hamiltonian. In general, if we have some arbitrary potential, around its minima it will look like a quadratic (you can see this from the Taylor series). Thus, we can approximate this potential around its minima as some small perturbation of the harmonic oscillator potential, which is quadratic.

Motivated by these examples, we will consider the general perturbed Hamiltonian $H = H^{(0)} + \delta H$, where $H^{(0)}$ is a well-understood Hamiltonian and $\delta H$ is a small perturbation. We are not going to formally define what it means to be ``small'' here; this turns out to be more complicated of a problem than it may seem.

\subsection{Non-Degenerate Perturbation Theory}
\subsection*{Perturbative Equations}
We would like to use series expansions to approach this problem, so we begin by adding a parameter $\lambda$ to the problem. We define $H(\lambda) := H^{(0)} + \lambda \delta H$, where $0<\lambda<1$. By varying $\lambda$ from $0$ to $1$, we will continuously turn on the perturbation.

We will now derive the perturbative equations. Note that we will assume that the original Hamiltonian has no degeneracies; that is, each energy has exactly one linearly independent eigenstate associated with it. This is because degeneracies will make our lives significantly more complicated as a perturbation may split a degeneracy, meaning that two states with the same energy may end up with different energies after the perturbation is applied. In 1 dimension, our assumption is fully general for bound states since degeneracies are prohibited.

Suppose our original Hamiltonian have eigenvalues $E_0^{(0)}, E_1^{(0)}, E_2^{(0)},...$ corresponding to eigenvectors $\ket{0^{(0)}}, \ket{1^{(0)}},\ket{2^{(0)}},... .$ We also assume that the energies are ordered, so $E_0^{(0)} < E_1^{(0)} < E_2^{(0)} < ... $. Let $\ket{n}_\lambda$ be the perturbed eigenstate beginning with energy $E_n^{(0)}$, and let $E_n (\lambda)$ be the perturbed energy. We will assume series expansions of the following form:
\begin{align*}
\ket{n}_{\lambda} &= \ket{n^{(0)}} + \lambda \ket{n^{(1)}} +  \lambda^2 \ket{n^{(2)}} + ..., \\
E_n (\lambda) &= E_n^{(0)} + \lambda E_n^{(1)} + \lambda^2 E_n^{(2)} + ... 
\end{align*}

We now apply the condition on the eigenstate,
\begin{align*}
&(H^{(0)} + \lambda \delta H) \left(\ket{n^{(0)}} + \lambda \ket{n^{(1)}} +  \lambda^2 \ket{n^{(2)}} + ... \right) = E_n (\lambda) \left(\ket{n^{(0)}} + \lambda \ket{n^{(1)}} +  \lambda^2 \ket{n^{(2)}} + ... \right) \\
&= (E_n^{(0)} + \lambda E_n^{(1)} + \lambda^2 E_n^{(2)} + ...)\left(\ket{n^{(0)}} + \lambda \ket{n^{(1)}} +  \lambda^2 \ket{n^{(2)}} + ... +  \lambda^k \ket{n^{(k)}}\right).
\end{align*}
Considering each side as a polynomial in $\lambda$, we match coefficients. 
\begin{align}
H^{(0)} \ket{n^{(0)}} &= E_n^{(0)} \ket{n^{(0)}}, \label{eq:0} \\
H^{(0)} \ket{n^{(1)}} + \delta H \ket{n^{(0)}} &= E_n^{(0)} \ket{n^{(1)}} + E_n^{(1)} \ket{n^{(0)}}, \label{eq:1} \\
H \ket{n^{(2)}} + \delta H \ket{n^{(1)}} &= E_n^{(0)} \ket{n^{(2)}} + E_n^{(1)} \ket{n^{(1)}} + E_n^{(2)} \ket{n^{(0)}}, \label{eq:2} \\
&...\nonumber
\end{align} In general,
\begin{equation} H^{(0)} \ket{n^{(j)}} + \delta H \ket{n^{(j-1)}} = \sum_{i=0}^{j} E_n^{(i)} \ket{n^{j-i}}.
\label{eq:gennondegen}
\end{equation}
\subsection*{First-Order Correction}
The first-order correction to the energy and eigenstate are easy to calculate. Without loss of generality, we assume that the different corrections to the eigenstate are orthogonal to the original state. We can do this because any components that overlap with $\ket{n^{(0)}}$ can be combined into the first term and divided out.

We act on both sides of the equation for the first-order correction to the eigenstate with $ \bra{n^{(0)}}$.
\begin{align}
\bra{n^{(0)}}H^{(0)} \ket{n^{(1)}} + \bra{n^{(0)}}\delta H \ket{n^{(0)}} &= \bra{n^{(0)}}E_n^{(0)} \ket{n^{(1)}} + \bra{n^{(0)}}E_n^{(1)}  \ket{n^{(0)}}, \nonumber \\
\bra{n^{(0)}}\delta H \ket{n^{(0)}} &= E_n^{(1)} , \label{eq:nondegen1}
\end{align}
where we used the Hermiticity of $H$ and the orthogonality of $\ket{n^{(0)}}$ and $\ket{n^{(1)}}$ to simplify. This result is incredibly useful because it allows us to calculate the first-order correction to the energy directly from $\ket{n^{(0)}}$ and $\delta H$, both of which are known.

We will now calculate the components of the first-order correction to the state in the original basis. We have 
\begin{align*} 
\bra{k^{(0)}}H^{(0)} \ket{n^{(1)}} + \bra{k^{(0)}}\delta H \ket{n^{(0)}} &= \bra{k^{(0)}}E_n^{(0)} \ket{n^{(1)}} + \bra{k^{(0)}}E_n^{(1)} \ket{n^{(0)}}, \\
(E_k^{(0)}-E_n^{(0)}) \braket{k^{(0)}}{n^{(1)}} &= - \bra{k^{(0)}} \delta H \ket{n^{(0)}} , \\
\braket{k^{(0)}}{n^{(1)}} &= -\frac{\bra{k^{(0)}} \delta H \ket{n^{(0)}}}{(E_k^{(0)}-E_n^{(0)})},
\end{align*}
where we assume that $k \neq n$ (note that if $k=n$, $\bra{k^{(0)}}\ket{n^{(1)}}$ is $0$ by construction).

Therefore,
\begin{align}
\ket{n^{(1)}} &= \sum_{k \neq n} \braket{k^{(0)}}{n^{(1)}}\ket{k^{(0)}}, \nonumber \\
&= -\sum_{k \neq n} \frac{\bra{k^{(0)}} \delta H \ket{n^{(0)}} \ket{k^{(0)}}}{(E_k^{(0)}-E_n^{(0)})}. \label{eq:nondegenn1}
\end{align}
This expression makes it clear why our assumption of non-degeneracy was necessary. If there were another state with energy $E_n^{(0)}$, the denominator of the corresponding term in the sum would go to $0$.
\subsection*{Problems}
\begin{enumerate}
\item Consider another anharmonic oscillator with the Hamiltonian $$H = H^{(0)} + \lambda \frac{m^2 \omega^3}{\hbar} \hat{x}^4.$$ Write the Hamiltonian in terms of $\hat{a}$ and $\hat{a}^{\dagger}$, then calculate the first-order correction in the ground-state energy. 
\item Using that fact that the expectation value of the Hamiltonian on any normalized state is larger than the ground state energy (the variational principle), prove that the first order corrected energy of the non-degenerate ground state overstates the actual exact ground state energy.
\end{enumerate}


\subsection {Degenerate Perturbation Theory}
We now consider the problem of degenerate perturbation theory. We will analyze the case where the degeneracy is lifted at first order; that is, we will assume that the first order correction is different for different degenerate states. The case where the degeneracy is not lifted at the first order is more complicated and will not be discussed here. 
\subsection* {Example}
To understand why a degeneracy presents a problem, we consider a simple example. Suppose \[H^{(0)} = \begin{pmatrix} 1 & 0 \\ 0 & 1 \end{pmatrix}\] and \[\delta H = \begin{pmatrix} 0 & 1 \\ 1 & 0 \end{pmatrix}.\] Then \begin{equation}
H (\lambda) = \begin{pmatrix} 1 & \lambda \\ \lambda & 1 \end{pmatrix}.
\end{equation}
One set of eigenvectors of $H^{(0)}$ is 
\[\ket{0^{(0)}} = \begin{pmatrix} 1 \\ 0 \end{pmatrix}, \ket{1^{(0)}} = \begin{pmatrix} 0 \\ 1 \end{pmatrix} \]
with the degenerate eigenvalue 1. 

We find, though, that when we solve for the eigenvectors and eigenvalues of $H(\lambda)$, we get eigenvectors 
$\frac{1}{\sqrt{2}}\begin{pmatrix}1\\1 \end{pmatrix}$ and $\frac{1}{\sqrt{2}}\begin{pmatrix}1\\-1 \end{pmatrix}$ with eigenvalues $1+\lambda$ and $1-\lambda$, respectively.

We might be tempted to think that, somehow, the system has "jumped" from one set of eigenvectors to another. What actually happened, though, is that we chose the "wrong" set of eigenvectors for the initial matrix. Because of the degeneracy, there is some ambiguity in which eigenvectors we choose. We have to choose the "right" basis to solve the problem effectively. 

\subsection* {Systematic Analysis}
Let us now derive general equations for a degeneracy that is lifted at first order.
We assume that there are N orthonormal degenerate eigenstates all with energy $E_n^{(0)}$, which we will denote by $\ket{n^{(0)}; k}$, where $k$ ranges from $1$ to $N$. Then we have \[\braket{n^{(0)};p}{n^{(0)};l} = \delta_{pl} \] and \[ H^{(0)} \ket{n^{(0)},k} = E_n^{(0)}. \] The degenerate eigenstates span a space $\mathbb{V}_N$ of dimension N, and the total state space $\mathcal{H}$ can be written as a direct sum $\mathcal{H} = \mathbb{V}_N \oplus \hat{V}$, where $\hat{V}$ is spanned by the eigenstates of $H^{(0)}$ not in $\mathbb{V}_N$.

We now consider the evolution of the degenerate eigenstates as the perturbation is turned on. We assume that $\ket{n^{(0)};k}$ evolves continuously to \[\ket{n;k}_\lambda = \ket{n^{(0)};k} + \lambda \ket{n^{(1)};k} + \lambda^2 \ket{n^{(2)};k} + \mathcal{O}(\lambda^3) \]
with energy 
\[ E_{n,k} (\lambda) = E_n^{(0)} + \lambda E_{n,k}^{(1)} + \lambda^2 E_{n,k}^{(2)} + \mathcal{O}(\lambda^3).\]
Substituting these forms into the energy eigenstate equation will give completely analogous equations to those in the nondegenerate case (equations \cref{eq:0,eq:1,eq:2}). We have
\begin{align}
H^{(0)} \ket{n^{(0)};k} &= E_n^{(0)} \ket{n^{(0)};k},  \label{eq:deg0}\\
H^{(0)} \ket{n^{(1)};k} + \delta H \ket{n^{(0)};k} &= E_n^{(0)} \ket{n^{(1)};k} + E_{n,k}^{(1)} \ket{n^{(0)};k},  \label{eq:deg1}\\
H^{(0)} \ket{n^{(2)};k} + \delta H \ket{n^{(1)};k} &= E_n^{(0)} \ket{n^{(2)};k} + E_{n,k}^{(1)} \ket{n^{(1)};k} + E_{n,k}^{(2)} \ket{n^{(0)};k},  \label{eq:deg2}\\
&... \nonumber
\end{align} 
As before, equation \ref{eq:deg0} is satisfied by definition. We now consider equation \ref{eq:deg1}. Let us act on both sides with $\bra{n^{(0)}; l}$, where $l \neq k$. We have 
\begin{align}
\bra{n^{(0)}; l}H^{(0)} \ket{n^{(1)};k}+\bra{n^{(0)}; l} \delta H \ket{n^{(0)};k} &= \bra{n^{(0)}; l}E_n^{(0)} \ket{n^{(1)};k}+\bra{n^{(0)}; l}E_{n,k}^{(1)} \ket{n^{(0)};k},\nonumber \\
E_n^{(0)} \braket{n^{(0)};l}{n^{(1)};k} + \bra{n^{(0)}; l} \delta H \ket{n^{(0)};k} &= E_n^{(0)} \braket{n^{(0)};l}{n^{(1)};k} + E_{n,k}^{(1)} \braket{n^{(0)};l}{n^{(0)};k},\nonumber \\
 \bra{n^{(0)}; l} \delta H \ket{n^{(0)};k} &= E_{n,k}^{(1)} \braket{n^{(0)};l}{n^{(0)};k}, \nonumber \\
 \bra{n^{(0)}; l} \delta H \ket{n^{(0)};k} &= E_{n,k}^{(1)} \delta_{lk}.\label{eq:deg3}
\end{align}
Equation \ref{eq:deg3} tells us that we \textit{must} choose the basis of degenerate states such that $\delta H$ is diagonal in the chosen basis. Note that this means $\delta H$ must be diagonal in $\mathbb{V}_N$, \textit{not} be diagonal on the whole space $\mathcal{H}$.
In addition, when $l = k$, equation \ref{eq:deg3} gives us the first-order correction to the energy:
\[\bra{n^{(0)}; k} \delta H \ket{n^{(0)};k} = E_{n,k}^{(1)} \delta_{lk}\label{eq:degEFirstOrder}.\] This is \textit{exactly} the same equation as in the nondegenerate case (see equation \ref{eq:nondegen1}). The difference is that in the degenerate case we must choose our basis states appropriately. 

In addition, we can't get the full first order correction to the state from equation \ref{eq:deg1} alone. We can get some information from equation \ref{eq:deg1} by acting with a nondegenerate state $\bra{p^{(0)}},$ where $p \neq n$, with unperturbed energy $E_p^{(0)}$. We find that
\begin{align}
\bra{p^{(0)}}H^{(0)} \ket{n^{(1)};k}+\bra{p^{(0)}} \delta H \ket{n^{(0)};k} &= \bra{p^{(0)}}E_n^{(0)} \ket{n^{(1)};k}+\bra{p^{(0)}}E_{n,k}^{(1)} \ket{n^{(0)};k},\nonumber \\
(E_n^{(0)}-E_p^{(0)}) \braket{p^{(0)}}{n^{(1)};k} &= \bra{p^{(0)}} \delta H \ket{n^{(0)};k} \nonumber \\
\braket{p^{(0)}}{n^{(1)};k} &=  \frac{1}{E_n^{(0)}-E_p^{(0)}}\bra{p^{(0)}} \delta H \ket{n^{(0)};k} \label{eq:deg1corr1}
\end{align}
Equation \ref{eq:deg1corr1} gives the component of $\ket{n^{(1)};k}$ in $\hat{V}$ as 
\begin{equation}
\ket{n^{(1)};k}\Bigg\rvert_{\hat{V}} = 
\sum\limits_{p \neq n} \frac{\bra{p^{(0)}} \delta H 
\ket{n^{(0)};k}}{E_n^{(0)}-E_p^{(0)}} \ket{p^{(0)}}\label{eq:deg1corr1s}.\end{equation}

This is \textit{all} the information we can extract from equation \ref{eq:deg1}. We have to use equation \ref{eq:deg2}, which involves the second-order terms, to derive the component of $\ket{n^{(1)};k}$ in $\mathbb{V}_N$. To do this, we hit equation \ref{eq:deg2} with $\bra{n^{(0)}; l}$ and find that
\begin{align*}
\bra{n^{(0)}; l}H^{(0)} \ket{n^{(2)};k} + \bra{n^{(0)}; l}\delta H \ket{n^{(1)};k} &= \bra{n^{(0)}; l}E_n^{(0)} \ket{n^{(2)};k} + \bra{n^{(0)}; l} E_{n,k}^{(1)} \ket{n^{(1)};k} + \bra{n^{(0)}; l} E_{n,k}^{(2)} \ket{n^{(0)};k}, \\
\bra{n^{(0)}; l}\delta H \ket{n^{(1)};k} &= E_{n,k}^{(1)} \braket{n^{(0)};l}{n^{(1)};k} + E_{n,k}^{(2)} \delta_{lk}.
\end{align*}
When $l \neq k$, this simplifies to 
\begin{equation}
\bra{n^{(0)}; l}\delta H \ket{n^{(1)};k} = E_{n,k}^{(1)} \braket{n^{(0)};l}{n^{(1)};k}. \label{idk}
\end{equation}
Recall that $\delta H$ is diagonal in $\mathbb{V}_N$, so the LHS of equation \ref{idk} simplifies to 
\begin{equation}
E_{n,l}^{(1)} \braket{n^{(0)};l}{n^{(1)};k} + \sum\limits_{p \neq n} \frac{\bra{p^{(0)}} \delta H 
\ket{n^{(0)};k}}{E_n^{(0)}-E_p^{(0)}} \bra{n^{(0)}; l}\delta H \ket{p^{(0)}}
 = E_{n,k}^{(1)} \braket{n^{(0)};l}{n^{(1)};k}. \label{ugly}.
\end{equation}
Equation \ref{ugly} can be solved to find the component of $\ket{n^{(1)};k}$ in $\mathbb{V}_N$. In the end, we find that

\begin{align*}
\ket{n;k}_\lambda &= \ket{n^{(0)};k}
- \lambda \left( \sum\limits_p \frac{\bra{p^{(0)}}\delta H \ket{n^{(0)};k}}{E_p^{(0)}-E_n^{(0)}}
+ \sum\limits_{\ell\neq k} \frac{\ket{n^{(0)};\ell}}{E_{n,k}^{(1)}-E_{n,\ell}^{(1)}} \sum\limits_{p} \frac{\bra{n^{(0)};\ell}\delta H \ket{p^{(0)}} \bra{p^{(0)}}\delta H\ket{n^{(0)};k}}{E_p^{(0)}-E_n^{(0)}}
\right) + \mathcal{O} (\lambda^3).
\end{align*}

This concludes our treatment of the first-order correction for degenerate perturbation theory in which the degeneracy is lifted at the first order.


