\section{Central Potentials}
We now move to 3 dimensions and consider central potentials. These are potentials of the form 
\[V(\vec{r}) = V(r). \]
In three dimensions, the Schrodinger equation becomes 
\begin{equation}
-\frac{\hbar^2}{2m} \laplacian{\psi} + V(\vec{r}) \psi  = E\psi 
\label{3DSchrodinger}.
\end{equation}
For central potentials, this becomes 
\begin{equation}
-\frac{\hbar^2}{2m} \laplacian{\psi} + V(r) \psi  = E\psi .
\end{equation}
It makes sense to work in spherical coordinates. Recall from multivariable calculus that the Laplacian in spherical coordinates is 
\begin{equation}
    \laplacian f = \frac{1}{r^2} \frac{\partial}{\partial r} \left(r^2 \frac{\partial f}{\partial r} \right) + \frac{1}{r^2 \sin \theta} \frac{\partial}{\partial \theta} \left(\sin \theta \frac{\partial f}{\partial \theta} \right) + \frac{1}{r^2 \sin^2 \theta} \frac{\partial^2 f}{\partial \phi^2},
\end{equation}
so the Schrodinger equation becomes 
\begin{equation}
    -\frac{\hbar^2}{2m} \left[\frac{1}{r^2} \frac{\partial}{\partial r} \left(r^2 \frac{\partial \psi}{\partial r} \right) + \frac{1}{r^2 \sin \theta} \frac{\partial}{\partial \theta} \left(\sin \theta \frac{\partial \psi}{\partial \theta} \right) + \frac{1}{r^2 \sin^2 \theta} \frac{\partial^2 \psi}{\partial \phi^2} \right] + V\psi = E\psi.
    \label{centralSchrodinger}
\end{equation}
We look for separable solutions of the form $\psi(r,\theta,\phi) = R(r) Y(\theta,\phi).$
Substituting into the Schrodinger equation (Equation \ref{centralSchrodinger}) gives 
\begin{align*}
      -\frac{\hbar^2}{2m} \left[\frac{Y}{r^2} \frac{d}{dr} \left(r^2 \frac{dR}{dr} \right) + \frac{R}{r^2 \sin \theta} \frac{\partial}{\partial \theta} \left(\sin \theta \frac{\partial Y}{\partial \theta} \right) + \frac{R}{r^2 \sin^2 \theta} \frac{\partial^2 Y}{\partial \phi^2} \right] + VRY &= ERY,\\
      \left[\frac{2mr^2}{\hbar^2} [E-V(r)] + \frac{1}{R} \frac{d}{dr} \left(r^2 \frac{dR}{dr} \right)\right] + \frac{1}{Y}\left[\frac{1}{\sin \theta} \frac{\partial}{\partial \theta} \left( \sin \theta \frac{\partial Y}{\partial \theta}\right)+\frac{1}{\sin^2 \theta} \frac{\partial^2 Y}{\partial \phi^2} \right]&= 0.
\end{align*}
The first term in brackets depends only on $r$, whereas the remaining term depends only on $\theta$ and $\phi$. Thus each term must be constant. We will write this constant as $\ell (\ell+1)$ for reasons that will soon be clear. We have 
\begin{align}
    \frac{2mr^2}{\hbar^2} [E-V(r)] + \frac{1}{R} \frac{d}{dr} \left(r^2 \frac{dR}{dr} \right) &= \ell(\ell+1),\label{radial1}\\
    \frac{1}{Y}\left[\frac{1}{\sin \theta} \frac{\partial}{\partial \theta} \left( \sin \theta \frac{\partial Y}{\partial \theta}\right)+\frac{1}{\sin^2 \theta} \frac{\partial^2 Y}{\partial \phi^2} \right] &= -\ell(\ell+1).\label{angular}
\end{align}
\subsection{Angular Equation}
\label{angeqn}
The angular dependence, which is contained in equation \ref{angular}, does not depend on the specific central potential being considered. 
We can multiply Equation \ref{angular} on both sides by $\sin^2 \theta Y$ to get 
\begin{equation}
    {\sin \theta} \frac{\partial}{\partial \theta} \left( \sin \theta \frac{\partial Y}{\partial \theta}\right)+ \frac{\partial^2 Y}{\partial \phi^2}  = -\ell(\ell+1)\sin^2 \theta Y.
\end{equation}
Trying separation of variables, we assume the form \begin{equation*}
    Y(\theta, \phi) = \Theta (\theta) \Phi (\phi).
    \end{equation*}
We then find that 
\begin{align*}
 \Phi \sin \theta \frac{d}{d\theta} \left(\sin\theta \frac{d\Theta}{d\theta} \right) + \Theta \frac{d^2 \Phi}{d\phi^2}&= -\ell (\ell+1) \sin^2 \theta \Theta \Phi,\\
 \left[\frac{1}{\Theta} \sin \theta \frac{d}{d\theta}\left(\sin \theta \frac{d\Theta}{d\theta} \right) + \ell(\ell+1) \sin^2 \theta \right] + \frac{1}{\Phi}\frac{d^2 \Phi}{d\phi^2} &= 0.
\end{align*}
Using the same idea as before, each term must be a constant. We let this constant be $m^2$:
\begin{align}
    \frac{1}{\Theta} \sin \theta \frac{d}{d\theta}\left(\sin \theta \frac{d\Theta}{d\theta} \right) + \ell(\ell+1) \sin^2 \theta &= m^2,\label{theta} \\
    \frac{1}{\Phi} \frac{d^2 \Phi}{d\phi^2} &= -m^2. \label{phi}
\end{align}
Equation \ref{phi} is trivial:
\begin{equation}
    \Phi(\phi) = e^{im\phi}.
\end{equation}
There are actually two solutions, but we cover both signs by letting $m$ be positive or negative.

We'll return to the equation in $\phi$ later.

Equation \ref{theta} is a bit harder. It turns out that the solution is 
\begin{equation}
    \Theta (\theta) = AP_\ell^m (\cos \theta),
\end{equation}
where $P_\ell^m$ is the \textbf{associated Legendre function}:
\begin{equation}
    P_\ell^m (x) = (1-x^2)^{|m|/2} \left(\frac{d}{dx} \right)^{|m|} P_\ell (x),
\end{equation}
and $P_\ell(x)$ is the $\ell$th Legendre polynomial. These are given by the \textbf{Rodrigues formula}:
\begin{equation}
    P_\ell (x) = \frac{1}{2^\ell \ell !} \left( \frac{d}{dx}\right) (x^2-1)^\ell.
\end{equation}
For the Rodrigues formula to make sense, $\ell$ must be a nonnegative integer. In addition, if $|m| > \ell$, the associated Legendre function must be zero. Thus $\ell$ must be an integer less or equal to $m$.

We can normalize the angular wavefunctions to get the \textbf{spherical harmonics}:
\begin{equation}
Y_\ell^m (\theta, \phi) = \epsilon \sqrt{\frac{(2\ell+1)}{4\pi} \frac{(\ell-|m|)!}{(\ell+|m|)!}} e^{im\phi} P_\ell^m (\cos \theta),
    \label{sphericalHarmonics}
\end{equation}
where $\epsilon = (-1)^m$ for positive $m$ and $\epsilon =1$ otherwise. The spherical harmonics are orthogonal.
\subsection{Radial Equation}
We can simplify the radial dependence contained in Equation \ref{radial1} by changing variables to $u(r) = r R(r)$. Then we have 
\begin{equation}
    -\frac{\hbar^2}{2m} \frac{d^2u}{dr^2} + \left[V+\frac{\hbar^2}{2m}\frac{\ell(\ell+1)}{r^2} \right] u  = Eu.\label{radial}
\end{equation}
Equation \ref{radial} is the \textbf{radial equation}. It looks very similar to the Schrodinger equation, except the potential has added the \textbf{centrifugal term} $\frac{\hbar^2}{2m}\frac{\ell(\ell+1)}{r^2}$.
\subsection{Angular Momentum}
Let's now revisit the angular equation from the perspective of angular momentum. Classically, angular momentum is given by \begin{equation*}
    \vec{L} = \vec{r} \times \vec{p}.
\end{equation*}
We can use this to construct the quantum mechanical angular momentum operators:
\begin{align}
    \hat{L}_x &= \hat{y} \hat{p}_z - \hat{z}\hat{p}_y = \frac{\hbar}{i} \left(y\frac{\partial}{\partial z} - z \frac{\partial}{\partial y} \right), \label{Lx}\\
    \hat{L}_y &= \hat{z} \hat{p}_x - \hat{x}\hat{p}_z= \frac{\hbar}{i} \left(z\frac{\partial}{\partial x} - x \frac{\partial}{\partial z} \right), \label{Ly}\\
    \hat{L}_z &= \hat{x} \hat{p}_y - \hat{y}\hat{p}_x
    = \frac{\hbar}{i} \left(x\frac{\partial}{\partial y} - y \frac{\partial}{\partial x} \right).\label{Lz}
\end{align}

\subsection*{Commutation Relations}
The angular momentum commutation relations are extremely significant. Using the definition in Equations \ref{Lx}-\ref{Lz}, it is easy to prove the following:
\begin{align}
    \commutator{\hat{L}_x}{\hat{L}_y} = i\hbar \hat{L}_z,\\
    \commutator{\hat{L}_y}{\hat{L}_z} = i\hbar \hat{L}_x,\\
    \commutator{\hat{L}_z}{\hat{L}_x} = i\hbar \hat{L}_y.
\end{align}
If we let $\hat{L}_1 = \hat{L}_x$, $\hat{L}_2 = \hat{L}_y$, and $\hat{L}_3 = \hat{L}_z$, the commutation relations can be written nicely in Einstein summation notation using the Levi-Civita tensor:
\begin{equation}
    \commutator{\hat{L}_i}{\hat{L}_j} = \epsilon_{ijk} i\hbar \hat{L}_k.
\end{equation}
It's also pretty simple to show that each of the individual angular momentum operators commutes with the square of the total angular momentum
\[\hat{L}^2 = \hat{L}_x^2 + \hat{L}_y^2 + \hat{L}_z^2. \]
This means we can find simultaneous eigenstates of $\hat{L}^2$ and $\hat{L}_z$.
\subsection*{Eigenvalues}
To find the eigenvalues and eigenstates of $\hat{L}^2$ and $\hat{L}_z$, we use ladder operators much like those we used in the harmonic oscillator. We define 
\begin{equation}
    \hat{L}_{\pm} \equiv \hat{L}_x \pm i\hat{L}_y.
\end{equation}
The commutators with $\hat{L}_z$ are
\begin{equation}
    \commutator{\hat{L}_z}{\hat{L}_{\pm}} = \commutator{\hat{L_z}}{\hat{L_x}} \pm i \commutator{\hat{L}_z}{\hat{L}_y} = i\hbar\hat{L}_y \pm \hbar \hat{L}_x = \pm \hbar \hat{L}_{\pm}.
\end{equation} Trivially, also, 
\begin{equation}
    \commutator{\hat{L}^2}{\hat{L}_\pm} = 0.
\end{equation}

Suppose $\ket{\lambda; \mu}$ is an eigenstate of $\hat{L}_z$ with eigenvalue $\mu$ and of $\hat{L}^2$ with eigenvalue $\lambda$. Now let's act on this state with the ladder operators and see what happens.

We have 
\begin{equation*}
    \hat{L}^2 (\hat{L}_\pm \ket{\lambda; \mu}) = \lambda \hat{L}_\pm \ket{\lambda; \mu}
\end{equation*}
and 
\begin{equation*}
    \hat{L}_z(\hat{L}_\pm \ket{\lambda; \mu}) = \commutator{\hat{L}_z}{\hat{L}_\pm} \ket{\lambda; \mu} + \hat{L}_\pm \hat{L}_z \ket{ \lambda; \mu} = (\mu \pm \hbar) \hat{L}_\pm \ket{\lambda; \mu}.
\end{equation*}
These equations show that the ladder operators maintain the same value of the total angular momentum and can increase or decrease the angular momentum in the $z$-direction in increments of $\hbar$. Thus, acting with them gives a ladder of states with the same total angular momentum.

Can we keep acting with these operators for as many times as we want? Well, at some point we'll reach a state for which the $z$-component is greater than the total angular momentum. Therefore, there must be a top state $\ket{\lambda; \mu_t}$ that is annihilated by $\hat{L}_+$ and a bottom state $\ket{\lambda; \mu_b}$ that is annihilated by $\hat{L}_-$.

Let the $z$-component of the angular momentum at the top be $\mu_t = \hbar \ell$ and at the bottom be $\mu_b = \hbar \tilde{\ell}$. 

We have 
\begin{equation*}
    \hat{L}_\pm \hat{L}_\mp = (\hat{L}_x \pm i\hat{L}_y) (\hat{L}_x \mp i \hat{L}_y) = \hat{L}_x^2 + \hat{L}_y^2 \pm i \commutator{\hat{L}_y}{\hat{L}_x} = \hat{L}_x^2 + \hat{L}_y^2 \pm \hbar \hat{L}_z,
\end{equation*}
so 
\begin{equation*}
    \hat{L}^2 = \hat{L}_\pm \hat{L}_\mp  \mp \hbar \hat{L}_z + \hat{L}_z^2.
\end{equation*}
Then 
\begin{equation*}
    \hat{L}^2 \ket{\lambda; \mu_t} = (\hat{L}_- \hat{L}_+ + \hbar \hat{L}_z + \hat{L}^2_z) \ket{\lambda; \mu_t} = \hbar^2 \ell (\ell + 1) \ket{\lambda; \mu_t},
\end{equation*} and
\begin{equation*}
    \hat{L}^2 \ket{\lambda; \mu_t} = (\hat{L}_+ \hat{L}_- - \hbar \hat{L}_z + \hat{L}^2_z) \ket{\lambda; \mu_t} = \hbar^2 \tilde{\ell} (\tilde{\ell} - 1) \ket{\lambda; \mu_b}.
\end{equation*}
Since these states have the same value of $\lambda$, we must have 
\begin{equation}\hbar^2 \tilde{\ell} (\tilde{\ell} - 1) = \hbar^2 \ell (\ell + 1),\end{equation} so \begin{equation}\tilde{\ell} = -\ell.\end{equation}

Clearly the eigenvalues of $\hat{L}_z$ are $m\hbar$, where $m$ ranges from $-\ell$ to $+\ell$ in integer steps. Therefore $\ell$ must be an integer or half-integer.

We can then characterize the eigenstates by two numbers, $\ell$ and $m$, such that the eigenvalue of the total angular momentum is $\hbar^2 \ell (\ell+1)$ and the eigenvalue of the $z$-component of the angular momentum is $\hbar m$.
\subsection*{Eigenfunctions}
To find the eigenfunctions of the angular momentum, we need to write the operators in spherical coordinates. It turns out that 
\begin{align}
    \hat{L}_x &= \frac{\hbar}{i} \left(-\sin \phi \frac{\partial}{\partial \theta} - \cos \phi \cot \theta \frac{\partial}{\partial \phi} \right), \\
    \hat{L}_y &= \frac{\hbar}{i} \left(+\cos \phi \frac{\partial}{\partial \theta} - \sin \phi \cot \theta \frac{\partial}{\partial \phi} \right), \\
    \hat{L}_z &= \frac{\hbar}{i} \frac{\partial}{\partial \phi},\\
    \hat{L}_{\pm} &= \pm \hbar e^{\pm i \phi} \left(\frac{\partial}{\partial \theta} \pm i \cot \theta \frac{\partial}{\partial \phi} \right).
\end{align}
Let $f_\ell^m$ be the eigenfunction corresponding to $\ell$ and $m$. The eigenvalue equation for $\hat{L}_z$ becomes 
\begin{equation*}
    \frac{\partial f}{\partial \phi} = i m f,
\end{equation*}
so 
\begin{equation*}
    f = g(\theta) e^{im\phi}.
\end{equation*}
After some significant simplification, the eigenvalue equation for $\hat{L}^2$ becomes
\begin{align}
         \sin \theta \frac{d}{d\theta}\left(\sin \theta \frac{dg}{d\theta} \right) + \ell(\ell+1) \sin^2 \theta g &= m^2g. \label{angular2}
\end{align}
Equation \ref{angular2} is equivalent to equation \ref{theta}. Therefore the simultaneous eigenfunctions of $\hat{L}^2$ and $\hat{L}_z$ are the spherical harmonics! 

Our analysis of angular momentum has actually given us some extra information about central potentials. It let us know the possible values for $\ell$ and $m$. One strange fact, though, is that our analysis from Section \ref{angeqn} forced $\ell$ to be an integer for classical angular momentum. The algebraic theory, though, allows for half-integer values of $\ell$. These end up being a new kind of angular momentum: spin.