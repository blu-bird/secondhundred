\section{Basics of Quantum Mechanics}
The experiments summarized in the previous section necessitated a new formulation of physics. We discuss the postulates of quantum mechanics in the next section. 
\subsection{Postulates of Quantum Mechanics}
These are some of the basic postulates of quantum mechanics:
\begin{postulate}
\label{post1}
The state of any particle is represented by a normalized vector $\ket{\psi}$ in a Hilbert space. 
\end{postulate}
\begin{postulate}
\label{post2}
To each observable $Q$ is associated a Hermitian operator $\hat{Q}$ for which the expectation value of $Q$ in that state is $\bra{\psi}\hat{Q}\ket{\psi}$.
\end{postulate}
\begin{postulate}
\label{post3}
When measuring the observable $Q$ on a particle whose state is $\ket{\psi}$, the only possible values are the eigenvalues of $\hat{Q}$. The probability of getting a certain eigenvalue $\lambda$, associated with normalized eigenvector $\ket{\lambda}$, is equal to $\norm{\braket{\lambda}{\psi}}^2$. After measurement, the state of the particle will collapse to the eigenvector of $\hat{Q}$ corresponding to the eigenvalue measured.
\end{postulate}
Note that the Hermiticity of the operators here is crucial. It guarantees that the eigenvalues, which are the possible results of a measurement, are real. It also means the eigenvectors are orthogonal, which means that, when the state of a particle is an eigenstate, only the eigenvalue associated with the state will be measured. In a finite-dimensional space, Hermiticity also implies that the eigenvectors of the operator span the  space. In an infinite-dimensional space, the operators corresponding to various observables must also satisfy this condition. This could be seen as another postulate. 

There are more postulates, which will be discussed in the next few sections. For now, these are the basic principles we need for our introduction to quantum mechanics. 
\subsection{The Wavefunction}
We're primarily going to work in the position basis in this set of lectures. This is an infinite-dimensional basis, since there are infinite possible positions. We'll work in one dimension for now.

We will work in the position basis. Let the eigenstate at position $x$ be $\ket{x}$. Suppose we have a particle in the state $\ket{\psi(t)} $ at time $t$. We define the \textbf{wavefunction} of this state in the position basis to be 
\begin{equation*}
\psi (x, t) = \braket{x}{\psi(t)}.
\end{equation*}
Since the position basis is orthonormal, we have 
\begin{equation}
\ket{\psi(t)} = \int\limits_{-\infty}^{\infty} \ket{x} \braket{x}{\psi(t)} dx = \int\limits_{-\infty}^{\infty} \ket{x} \psi(x, t) dx.
\end{equation}

The wavefunction gives the probability density of being found at a certain position. The probability that a particle in state $\ket{\psi(t)}$ is found between $x$ and $x+dx$ is 
\begin{equation*}
\norm{\braket{x}{\psi(t)}}^2 dx = \norm{\psi(x,t)}^2 dx,
\end{equation*}
so the probability that it is found between $x=a$ and $x=b$ is 
\begin{equation*}
\int \limits_{a}^{b} \norm{\psi(x,t)}^2 dx.
\end{equation*}
Since the particle has to be found \textit{somewhere}, we must have 
\begin{equation}
\int \limits_{-\infty}^{\infty} \norm{\psi(x,t)}^2 dx = 1.
\label{normalization}
\end{equation}
Equation \ref{normalization} is the normalization condition.

The expectation value of some operator $\hat{Q}$ is 
\begin{equation*}
\bra{\psi(t)} \hat{Q} \ket{\psi(t)} = \int\limits_{-\infty}^{\infty} \braket{\psi(t)}{x}  \braket{x}{\hat{Q} \psi(t)} dx = \int\limits_{-\infty}^{\infty} \psi^*(x,t) \hat{Q} \psi(x,t) dx.
\end{equation*}

The position operator $\hat{x}$ must satisfy 
\begin{equation*}
\hat{x} \ket{x'} = x' \ket{x'}.
\end{equation*}
Therefore 
\begin{equation*}
\hat{x} \ket{\psi(t)} = \int\limits_{-\infty}^{\infty} \hat{x} \ket{x} \psi(x,t) dx = \int\limits_{-\infty}^{\infty} \ket{x} x \psi(x,t) dx,
\end{equation*}
so 
\begin{equation}
\hat{x} \psi(x,t) = x \psi(x,t).
\end{equation}
\subsection{Schrodinger's Equation and Time Evolution}
The time evolution of the wavefunction is given by the \textbf{Schrodinger equation}:
\begin{equation}
\label{schrodinger}
i \hbar \frac{\partial \psi}{\partial t} = -\frac{\hbar^2}{2m} \frac{\partial^2 \psi}{\partial x^2} + V(x) \psi.
\end{equation}
$V(x)$ is the potential as a function of position. We assume that it is time-independent. Equation \ref{schrodinger} is the quantum-mechanical equivalent of Newton's law: given some initial state, it determines the state for all future times. The operator 
\begin{equation}
\hat{H} =  -\frac{\hbar^2}{2m} \frac{\partial^2 }{\partial x^2} + V(\hat{x})
\label{hamiltonian}
\end{equation}
is the Hamiltonian.

We now define a time evolution operator $\hat{U} (t)$ such that 
\begin{equation*}
\hat{U} (t) \ket{\psi (0)} = \ket{\psi(t)}.
\end{equation*} 

In general, substituting into Equation \ref{schrodinger} gives 
\begin{align*}
i \hbar \frac{\partial \psi (x,t)}{\partial t} &= \hat{H} \psi (x,t),\\
i \hbar \frac{\partial \hat{U} \psi (x,0)}{\partial t} &= \hat{H}\hat{U} \psi (x,0),\\
\left(i\hbar\frac{\partial \hat{U}}{\partial t} - \hat{H}\hat{U} \right) \psi (x,0) &= 0,\\
i\hbar\frac{\partial \hat{U}}{\partial t} - \hat{H}\hat{U} &= \hat{0}, \\
i\hbar \frac{\partial \hat{U}}{\partial t} &= \hat{H}\hat{U}. \\
\end{align*}
Taking the adjoint of both sides of this equation, we find
\begin{equation*}
-i\hbar \frac{\partial \hat{U}^{\dagger}}{\partial t} = \hat{U}^{\dagger} \hat{H}.
\end{equation*}
Then 
\begin{align*}
\frac{d}{dt} (\hat{U}^{\dagger}\hat{U}) &= \frac{d\hat{U}^{\dagger}}{dt}\hat{U} + \hat{U}^{\dagger} \frac{d\hat{U}}{dt}, \\
&= \frac{i}{\hbar} \left(\hat{U}^{\dagger} \hat{H} \hat{U} - \hat{U}^{\dagger} \hat{H} \hat{U} \right) \\
&= 0.
\end{align*}
Since 
\begin{equation*}
\hat{U}(0) = \hat{I},
\end{equation*}
we have
\begin{equation}
\hat{U}^{\dagger}\hat{U} = \hat{I}
\end{equation}
at all times. Thus time evolution is unitary.

This means that states maintain their normalization over time:
\begin{equation*}
\braket{\psi(t)}{\psi(t)} = (\hat{U} (t) \ket{\psi (0)})^{\dagger} \hat{U} (t) \ket{\psi (0)}
= \bra{\psi(0)} \hat{U}^{\dagger}(t)\hat{U}(t) \ket{\psi(0)} = 1.
\end{equation*}

For time-independent Hamiltonians, we also find that 
\begin{equation} \hat{U}(t) = e^{-\frac{i\hat{H} t}{\hbar}}. \end{equation}
\subsection{Momentum}
The expectation value of the position is 
\begin{equation}
\bra{\psi(t)}\hat{x}\ket{\psi(t)} =  \int\limits_{-\infty}^{\infty} \psi^*(x,t) x \psi(x,t) dx.
\end{equation} 
To understand momentum, we consider the change in the expectation value of position over time. We have 
\begin{align*}
\frac{d \langle \hat{x} \rangle}{dt} &= \int\limits_{-\infty}^{\infty} x\left(\frac{d\psi^*}{dt}  \psi  +\psi^*  \frac{d\psi}{dt}\right) dx,\\
\psi \frac{d\psi^*}{dt} &= \psi \frac{i}{\hbar} \left( -\frac{\hbar^2}{2m} \frac{\partial^2 \psi^*}{\partial x^2} + V(x) \psi^* \right),\\
\psi^* \frac{d\psi}{dt} &= -\psi^* \frac{i}{\hbar} \left( -\frac{\hbar^2}{2m} \frac{\partial^2 \psi}{\partial x^2} + V(x) \psi \right),\\
\frac{d \langle \hat{x} \rangle}{dt} &= \int\limits_{-\infty}^{\infty} x\frac{i\hbar}{2m} \left(\psi^*\frac{\partial^2 \psi}{\partial x^2}-\psi\frac{\partial^2 \psi^*}{\partial x^2} \right) dx,\\
&=  \int\limits_{-\infty}^{\infty} x\frac{i\hbar}{2m} \frac{\partial}{\partial x} \left(\psi^*\frac{\partial \psi}{\partial x}-\psi\frac{\partial \psi^*}{\partial x} \right)dx, \\
&= \frac{i\hbar}{2m} \left[x\left(\psi^*\frac{\partial \psi}{\partial x}-\psi\frac{\partial \psi^*}{\partial x} \right) \Bigg|_{-\infty}^{\infty} -  \int\limits_{-\infty}^{\infty} \left(\psi^*\frac{\partial \psi}{\partial x}-\psi\frac{\partial \psi^*}{\partial x} \right)dx \right],\\
&= -\frac{i\hbar}{2m}  \int\limits_{-\infty}^{\infty} \left(\psi^*\frac{\partial \psi}{\partial x}-\psi\frac{\partial \psi^*}{\partial x} \right)dx,\\
&= -\frac{i\hbar}{2m} \left[\int\limits_{-\infty}^{\infty} \psi^*\frac{\partial \psi}{\partial x}dx-\left(\psi \psi^* \Bigg|_{-\infty}^{\infty}-\int\limits_{-\infty}^{\infty}\psi^*\frac{\partial \psi}{\partial x} dx\right)\right],\\
&= -\frac{i\hbar}{m} \int\limits_{-\infty}^{\infty} \psi^*\frac{\partial \psi}{\partial x}dx.
\end{align*}

We'd expect the expectation value of momentum to relate to the expectation value of position by the following relation:
\begin{equation}
\langle \hat{p} \rangle = m \frac{d\langle \hat{x} \rangle}{dt}.
\end{equation}
Therefore 
\begin{equation*}
\langle \hat{p} \rangle =  -i\hbar \int\limits_{-\infty}^{\infty} \psi^*\frac{\partial \psi}{\partial x}dx,
\end{equation*}
so 
\begin{equation}
\hat{p} = -i\hbar \frac{d}{dx}.
\end{equation}
This is the momentum operator. 
\subsection{Energy}
We can now construct the (non-relativistic) energy operator. Classically, we have
$$E =  \frac{1}{2} mv^2 + V(x) = \frac{p^2}{2m} + V(x).$$ We can therefore construct the energy operator as 
$$\hat{E} =  \frac{\hat{p}^2}{2m} + V(\hat{x}) = -\frac{\hbar^2}{2m} \frac{\partial^2 }{\partial x^2} + V(\hat{x}) = \hat{H} $$ as expected. 

Let's now work in the energy basis. Let $\ket{\phi_E (t)}$ be the energy eigenstate with eigenvalue $E$ at time $t=0$. The Schrodinger equation gives 
\[i\hbar\frac{\partial}{\partial t} \ket{\phi_E} = \hat{H} \ket{E} = E \ket{\phi_E},\]
so 
\begin{equation}
\ket{\phi_E (t)} = e^{-\frac{iEt}{\hbar}} \ket{\phi_E(0)}.
\label{energyTimeEvolution}
\end{equation}
The energy eigenstate \textit{stays} an energy eigenstate with the same eigenvalue. It evolves only by an overall phase, which is immaterial. For this reason, energy eigenstates are called \textbf{stationary states}.

Equation \ref{energyTimeEvolution} allows us to evolve any state over time as long as we're working in the energy basis. Suppose we have some general state $$\ket{\psi(0)} = \sum_{n} c_n \ket{E_n},$$ where $\ket{E_n}$ is an energy eigenstate with energy $E_n$. Then
\begin{equation}
\ket{\psi(t)} = \sum_{n} c_n e^{-\frac{iE_nt}{\hbar}} \ket{E_n}.
\end{equation}
This means that solving the time-dependent Schrodinger equation is equivalent to solving for the energy eigenstates. For this reason, the energy eigenvalue equation is known as the \textbf{time-independent Schrodinger equation}. The time-independent equation is
\begin{equation}
-\frac{\hbar^2}{2m} \frac{d^2 \psi}{d x^2} + V(x) \psi = E \psi.
\label{timeIndSchrodinger}
\end{equation}
Note that we no longer have to solve a partial differential equation. Equation \ref{timeIndSchrodinger} can still only be solved exactly in a few cases.
\subsection{Generalized Uncertainty Principle}
We define the uncertainty of some operator $\hat{A}$ in state $\ket{\psi}$ to be 
\begin{equation}
\Delta A = \sqrt{\langle \hat{A}^2 \rangle  - \langle \hat{A} \rangle^2}.
\end{equation}

We can write 
\begin{equation*}
(\Delta A)^2 = \langle \hat{A}^2 \rangle  - \langle \hat{A} \rangle^2 = \braket {(\hat{A} - \langle A \rangle ) \psi} {(\hat{A} - \langle A \rangle ) \psi} = \braket {f}{f},
\end{equation*}
where $f = (\hat{A} - \langle A \rangle ) \psi$. For another operator $\hat{B}$, we can write 
\begin{equation*}
(\Delta B)^2 = \braket{g}{g}
\end{equation*}
for $g = (\hat{B} - \langle B \rangle ) \psi$.

Using the Schwarz inequality, we get 
\begin{align}
(\Delta A)^2(\Delta B)^2 &= \braket{f}{f} \braket{g}{g} \geq \norm{\braket{f}{g}}^2,\nonumber\\
&\geq (\operatorname{Im} (\braket{f}{g}))^2 = \left[\frac{1}{2i}\left(\braket{f}{g}-\braket{g}{f} \right) \right]^2,\nonumber\\
\braket{f}{g} &= \braket{(\hat{A} - \langle A \rangle ) \psi}{(\hat{B} - \langle B \rangle ) \psi}, \\
&= \expvalue{\hat{A}\hat{B}} + \expvalue{A}\expvalue{B}, \nonumber \\
\braket{g}{f} &=  \expvalue{\hat{B}\hat{A}} + \expvalue{A}\expvalue{B},\nonumber \\
(\Delta A)^2(\Delta B)^2 &\geq \left(\frac{1}{2i} \expvalue{\commutator{\hat{A}}{\hat{B}}}\right)^2. \label{uncertainty}
\end{align}
Equation \ref{uncertainty} is the \textbf{generalized uncertainty principle}. 

In the case that the operators in question are position and momentum, we find
\begin{equation}
\Delta x \Delta p \geq \frac{\hbar}{2}.
\end{equation}
This is the well-known \textbf{Heisenberg uncertainty principle}.

In general, operators that do not commute correspond to \textbf{incompatible observables} for which there is some uncertainty principle: you cannot know the values of both simultaneously to arbitrarily low precision. 