\section{The End?}
% Even though your knowledge of physics might just be constrained to what's in this document, this isn't even close to the modern fields of research in physics. Classical physics by itself is a very wide field, and we've only covered the most important parts of mechanics and electromagnetism that are considered to be testable on the AP Physics exam and class (with some other things). In what's considered "classical physics," we haven't even looked at the transfer of energy and heat, the heart of the field of thermodynamics (that you may have begun to look at in chemistry). Neither have we looked at the physics of waves and oscillations, which AP Physics skips over even though Physics 1 looks at them (ahem wave labs). Also, there are formulations of mechanics (by Lagrange and Hamilton) that are helpful when Newton's Laws are just not useful for analyzing the system, even with the use of conserved quantities. \\ \linebreak
% Furthermore, once Maxwell discovered that light was a wave, the system of classical physics that had been built on fundamental principles began to fall apart. Waves, as scientists of the time knew, had to travel through some sort of medium. For example, sound waves travel through air, tidal waves travel through water, etc. It was hypothesized that light (and other electromagnetic waves) traveled through a substance called the luminiferous aether at a speed $c$. However, this motion is relative, so if an observer is moving relative to the aether, the light's measured speed should be dependent upon the direction of the observer's motion, because of our principles of relative motion. However, this was refuted by the Michelson-Morley experiment, which showed that light traveled at the same speeds in all directions, and implied that aether was non-existent. This was a crisis - Maxwell's theory of electromagnetism appeared to contradict Newtonian mechanics - only one of these two theories can be absolutely correct. \\ \linebreak
% Enter Albert Einstein. He believed that Maxwell's theory of electromagnetism was correct and what we believed about space and time in Newtonian mechanics was just plain wrong. By imposing the speed of light as a universal speed limit, this means that when observers are moving, their notion of time and space changes as well relative to a stationary observer. This was his principle of special relativity. Ten years later, he would also publish his theory of general relativity, stating that large quantities of mass warped space and time, creating a gravitational force that could bend light.\\\linebreak
% Furthermore, the double-slit experiment for light showed that light exhibited wave-like properties of interference and diffraction, but if observed, the light appeared to exhibit properties of particles. This was called the wave-particle duality of light and showed the uncertainty present in measurements at the atomic level and smaller. This laid the foundations of quantum mechanics, a completely new branch of physics dealing with probability and the dynamics of small particles. \\ \linebreak
% So is physics solved or finished? Not by a long shot. One of the current great problems in modern physics is the formulation of the Theory of Everything, which would serve to reconcile the explanation of gravity due to general relativity with the quantum description of the electromagnetic force (and two other fundamental forces, the strong and weak nuclear forces). There's still so many things to do, questions to answer, and things to discover. If this introduction to physics has piqued your interest, there are many paths to explore and study still.
\pagebreak